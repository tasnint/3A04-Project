\documentclass[]{article}

% Imported Packages
%------------------------------------------------------------------------------
\usepackage{amssymb}
\usepackage{amstext}
\usepackage{amsthm}
\usepackage{amsmath}
\usepackage{enumerate}
\usepackage{fancyhdr}
\usepackage[margin=1in]{geometry}
\usepackage{graphicx}
\usepackage{extarrows}
\usepackage{setspace}
\usepackage{float}
%------------------------------------------------------------------------------

% Header and Footer
%------------------------------------------------------------------------------
\pagestyle{plain}  
\renewcommand\headrulewidth{0.4pt}                                      
\renewcommand\footrulewidth{0.4pt}                                    
%------------------------------------------------------------------------------

% Title Details
%------------------------------------------------------------------------------
\title{Deliverable \#3: What’s That Dish Software}
\author{SE 3A04: Software Design II -- Large System Design}
\date{}                               
%------------------------------------------------------------------------------

% Document
%------------------------------------------------------------------------------
\begin{document}

\maketitle	
\noindent{\bf Tutorial Number:} T03\\
{\bf Group Number:} G03 \\
{\bf Group Members:} 
\begin{itemize}
	\item Imran Chowdhury
	\item Michael Roberts
	\item Sathurshan Arulmohan
	\item Tanisha Tasnin
	\item Zifan Si
\end{itemize}

\section{Introduction}
\label{sec:introduction}
% Begin Section

This section should provide an brief overview of the entire document.

\subsection{Purpose}
\label{sub:purpose}
% Begin SubSection
\begin{enumerate}[a)]
	\item Delineate the purpose of the document
	\item Specify the intended audience for the document
\end{enumerate}
% End SubSection

\subsection{System Description}
\label{sub:system_description}
% Begin SubSection
\begin{enumerate}[a)]
	\item Give a brief description of the system. This could be a paragraph or two to give some context to this document.
\end{enumerate}
% End SubSection

\subsection{Overview}
\label{sub:overview}
% Begin SubSection
The document will present the state diagram of each controller from the class analysis diagram in section 2.
Section 3 will provide sequence diagrams for each use case of What's That Dish.
Finally section 4 shows the detailed class diagram of the application.

% End SubSection

% End Section

\section{State Charts for Controller Classes}
\label{sec:state_charts_for_controller_classes}
% Begin Section
\begin{figure}[H]
	\centering
   \includegraphics[width=\textwidth]{image/D3_state_diagrams/dish_determination.png}
\end{figure}

\begin{figure}[H]
	\centering
   \includegraphics[width=\textwidth]{image/D3_state_diagrams/image_recognition.png}
\end{figure}

\begin{figure}[H]
	\centering
   \includegraphics[width=\textwidth]{image/D3_state_diagrams/ingredients_analysis.png}
\end{figure}


\begin{figure}[H]
	\centering
   \includegraphics[width=\textwidth]{image/D3_state_diagrams/text_analysis.png}
\end{figure}

\begin{figure}[H]
	\centering
   \includegraphics[width=\textwidth]{image/D3_state_diagrams/account_management.png}
\end{figure}

\begin{figure}[H]
	\centering
   \includegraphics[width=\textwidth]{image/D3_state_diagrams/recommendation_system.png}
\end{figure}

\begin{figure}[H]
	\centering
   \includegraphics[width=\textwidth]{image/D3_state_diagrams/recipe_management.png}
\end{figure}

% End Section

\section{Sequence Diagrams}
\label{sec:sequence_diagrams}
% Begin Section
This section should provide a sequence diagram for each use case of your application.
% End Section

\section{Detailed Class Diagram}
\label{sec:detailed_class_diagram}
% Begin Section
\begin{figure}[H]
	\centering
   \includegraphics[width=\textwidth]{image/detailedClassDiagram.png}
\end{figure}

% End Section

\appendix
\section{Division of Labour}
\label{sec:division_of_labour}
% Begin Section
\textbf{Imran Chowdhury:}
\begin{enumerate}
	\item TODO
\end{enumerate}

\textbf{Signature:} Imran Chowdhury \\

\textbf{Michael Roberts:}
\begin{enumerate}
	\item TODO
\end{enumerate}

\begin{figure}[H]
 	\centering
    \includegraphics[width=\textwidth]{image/A_Michael_Roberts_Signature.png}
\end{figure}

\textbf{Sathurshan Arulmohan:}
\begin{enumerate}
	\item Developed state diagrams for recipe management, image recognition, text analysis, and ingredients analysis controllers.
	\item Worked with Tanisha on state diagram for dish determination.
	\item Reviewed and provided feedback for state diagrams for recommendation system and account managerment controller.
	\item Edited the detailed class diagram.
	\item Wrote section 1.3.
\end{enumerate}

\textbf{Signature:} SATHURSHAN ARULMOHAN \\

\textbf{Tanisha Tasnin:}
\begin{enumerate}
	\item TODO
\end{enumerate}

\textbf{Signature:} TANISHA TASNIN \\

\textbf{Zifan Si:}
\begin{enumerate}
	\item Worked on part 4 class diagram.
	\item add fix to D2 3.2 based on feedback.
\end{enumerate}

\textbf{Signature:} ZIFAN SI  \\
% End Section

\newpage
\section*{IMPORTANT NOTES}
\begin{itemize}
	\item You do \underline{NOT} need to provide a text explanation of each diagram; the diagram should speak for itself
	\item Please document any non-standard notations that you may have used
	\begin{itemize}
		\item \emph{Rule of Thumb}: if you feel there is any doubt surrounding the meaning of your notations, document them
	\end{itemize}
	\item Some diagrams may be difficult to fit into one page
	\begin{itemize}
		\item It is OK if the text is small but please ensure that it is readable when printed
		\item If you need to break a diagram onto multiple pages, please adopt a system of doing so and throughly explain how it can be reconnected from one page to the next; if you are unsure about this, please ask me
	\end{itemize}
	\item Please submit the latest version of Deliverable 1 and Deliverable 2 with Deliverable 3
	\begin{itemize}
		\item They do not have to be a freshly printed versions; the latest marked versions are OK
	\end{itemize}
	\item If you do \underline{NOT} have a Division of Labour sheet, your deliverable will \underline{NOT} be marked
\end{itemize}


\end{document}
%------------------------------------------------------------------------------