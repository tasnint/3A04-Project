\documentclass[]{article}

% Imported Packages
%------------------------------------------------------------------------------
\usepackage{amssymb}
\usepackage{amstext}
\usepackage{amsthm}
\usepackage{amsmath}
\usepackage{enumerate}
\usepackage{fancyhdr}
\usepackage[margin=1in]{geometry}
\usepackage{graphicx}
%\usepackage{extarrows}
%\usepackage{setspace}
%\usepackage{xcolor}
\usepackage{color}
%------------------------------------------------------------------------------

% Header and Footer
%------------------------------------------------------------------------------
\pagestyle{plain}  
\renewcommand\headrulewidth{0.4pt}                                      
\renewcommand\footrulewidth{0.4pt}                                    
%------------------------------------------------------------------------------

% Title Details
%------------------------------------------------------------------------------
\title{Deliverable \#1 Template : Software Requirement Specification (SRS)}
\author{SE 3A04: Software Design II -- Large System Design}
\date{}
                            
%------------------------------------------------------------------------------

% Document
%------------------------------------------------------------------------------
\begin{document}

\maketitle	
\noindent{\bf Tutorial Number:} T03\\
{\bf Group Number:} G03 \\
{\bf Group Members:} 
\begin{itemize}
	\item Imran Chowdhury
	\item Michael Roberts
	\item Sathurshan Arulmohan
	\item Tanisha Tasnin
	\item Zifan Si
\end{itemize}

\section*{IMPORTANT NOTES}
\begin{itemize}
	\item Be sure to include all sections of the template in your document regardless whether you have something to write for each or not
	\begin{itemize}
		\item If you do not have anything to write in a section, indicate this by the \emph{N/A}, \emph{void}, \emph{none}, etc.
	\end{itemize}
	\item Uniquely number each of your requirements for easy identification and cross-referencing
	\item Highlight terms that are defined in Section~1.3 (\textbf{Definitions, Acronyms, and Abbreviations}) with \textbf{bold}, \emph{italic} or \underline{underline}
	\item For Deliverable 1, please highlight, in some fashion, all (you may have more than one) creative and innovative features. Your creative and innovative features will generally be described in Section~2.2 (\textbf{Product Functions}), but it will depend on the type of creative or innovative features you are including.
\end{itemize}

\newpage
\section{Introduction}
\label{sec:introduction}
% Begin Section

\begin{itemize}
	\item Provide an overview of the document/SRS.
\end{itemize}


\subsection{Purpose}
\label{sub:purpose}
% Begin SubSection
\begin{itemize}
	\item Specify the purpose of the SRS.
	\item Specify the intended audience for the SRS.
\end{itemize}
% End SubSection

\subsection{Scope}
\label{sub:scope}
% Begin SubSection
\begin{itemize}
	\item Identify the software product(s) to be produced, and name each (e.g., Host DBMS, Report Generator, etc.)
	\item Explain what the software product(s) will do (and, if necessary, also state what they will not do).
	\item Describe the application of the software being specified, including relevant benefits, objectives, and goals.
%	\item Be consistent with similar statements in higher-level specifications (e.g., the system requirements specification), if they exist
\end{itemize}
% End SubSection

\subsection{Definitions, Acronyms, and Abbreviations}
\label{sub:definitions_acronyms_and_abbreviations}
% Begin SubSection
\begin{itemize}
	\item Provide the definitions of all terms, acronyms, and abbreviations required to properly interpret the SRS.
	\item This should be in alphabetical order.
\end{itemize}
% End SubSection

\subsection{References}
\label{sub:references}
% Begin SubSection
\begin{itemize}
	\item Provide a complete list of all documents referenced elsewhere in the SRS.
	\item Identify each document by title, report number (if applicable), date, and publishing organization.
	\item Specify the sources from which the references can be obtained.
	\item Order this list in some sensible manner (alphabetical by author, or something else that makes more sense).
\end{itemize}

\bibliographystyle{IEEEtran}
\renewcommand{\refname}{}  % Remove "References" title
\vspace{-7mm}  % Adjust spacing before bibliography
\bibliography{references}
% End SubSection

\subsection{Overview}
\label{sub:overview}
% Begin SubSection
\begin{itemize}
	\item Describe what the remainder of the document/SRS contains.\\
	(e.g. "Section 2 discusses...Section 3...")
%	\item Explain how the SRS is organized
\end{itemize}
% End SubSection

% End Section

\section{Overall Product Description}
\label{sec:overall_description}
% Begin Section

\begin{itemize}
	\item This section should describe the general factors that affect the product and its requirements. 
	\item It does not state specific requirements.
	\item It provides a \emph{background} for those requirements and makes them easier to understand.
\end{itemize}


\subsection{Product Perspective}
\label{sub:product_perspective}
% Begin SubSection
\begin{itemize}
	\item Put the product into perspective with other related products, i.e., context
	\item If the product is independent and totally self-contained, it should be stated here
	\item If the SRS defines a product that is a component of a larger system, then this subsection should relate the requirements of that larger system to the functionality of the software being developed. Identify interfaces between that larger system and the software to be developed.
	\item A block diagram showing the major components of the larger system, interconnections, and external interfaces can be helpful
\end{itemize}
% End SubSection

\subsection{Product Functions}
\label{sub:product_functions}
% Begin SubSection
\begin{itemize}
	\item Provide a \emph{summary} of the major functions that the software will perform.
	\begin{itemize}
		\item \textbf{Example}: An SRS for an accounting program may use this part to address customer account maintenance, customer statement, and invoice preparation without mentioning the vast amount of detail that each of those functions requires.
	\end{itemize}
	\item Functions should be organized in a way that makes the list of functions understandable to the customer or to anyone else reading the document for the first time 
	\item Present the functions in a list format - each item should be one function, with a brief description of it
	\item Textual or graphical methods can be used to show the different functions and their relationships
	\begin{itemize}
		\item Such a diagram is not intended to show a design of a product, but simply shows the logical relationships among variables
	\end{itemize} 
\end{itemize}
% End SubSection

\subsection{User Characteristics}
\label{sub:user_characteristics}
% Begin SubSection
\begin{itemize}
	\item Describe those general characteristics of the intended users of the product including educational level, experience, and technical expertise 
	\item Since there will be many users, you may wish to divide into different user types or personas
%	\item Do not state specific requirements, but rather provide the reasons why certain specific requirements are later specified
\end{itemize}
% End SubSection

\subsection{Constraints}
\label{sub:constraints}
% Begin SubSection
\begin{enumerate}
	\item \textbf{Data and Storage Constraints: }Data submitted by the user must be stored locally or in a cloud database, following Android data storage best practices. Storage limitations must be considered when handling high-resolution images to prevent excessive memory usage. If a cloud-based solution is used, the system must comply with API rate limits and data quotas.
	\item \textbf{Budget Constraints: }The project has a limited budget, which impacts the choice of third-party APIs, storage solutions, and computational resources. This means that free or open-source image recognition and NLP models should be prioritized to minimize costs.
	\item \textbf{Timeline Constraints: }The system must be developed within the allocated course duration, limiting the scope of certain advanced features such as restaurant API integration pr expanded multilingual support may need to be postponed to future versions.
\end{enumerate}
% End SubSection

\subsection{Assumptions and Dependencies}
\label{sub:assumptions_and_dependencies}
% Begin SubSection
Assumptions made in interpreting what the software being developed is aiming to achieve:
	\begin{enumerate}
		\item The user will provide clear and distinguishable images of food items to enable accurate identification by the Image Recognition Expert.
		\item The Text Description Expert assumes that users will provide detailed descriptions of food items, including key attributes such as color, texture, or preparation method.
		\item The Ingredient-Based Expert assumes that users will provide a complete and accurate ingredient list for identification to work correctly.
		\item Users will have an active internet connection when interacting with the system for real-time processing and recommendations.
		\item The personalized recommendation system assumes that users submit multiple food items over time to generate tailored suggestions.
		\item The application assumes that it will operate on Android devices with a minimum required processing power and storage space to handle food image recognition and database queries.
	\end{enumerate}

Any other assumptions made that, if it fails to hold, could require you to change the requirements:
	%\item List each of the factors that affect the requirements stated in the SRS
	%\item These factors are not design constraints on the software but are, rather, any changes to them that can affect the requirements in the SRS
	\begin{enumerate}
		\item The system shall support the latest version of the Android operating system.
		\item Assume that the cost, availability, and accessibility of third-party APIs will remain unchanged.
		\item The system complies with the latest version of GDPR, PIPEDA, and other data protection regulations.
		\item The food database used for identification is accurate, regularly updated, and includes a diverse range of cuisines.
	\end{enumerate}

% End SubSection

\subsection{Apportioning of Requirements}
\label{sub:apportioning_of_requirements}
% Begin SubSection
\begin{enumerate}
    \item \textbf{Multilingual Support: } The initial version of the system will support English only, with additional language options planned for future updates.
    \item \textbf{Offline Mode: } Future versions will introduce offline functionality with local storage or caching for limited food recognition.
    \item \textbf{Integration with Third-Party Applications: } The system will expand to integrate with food-related services, including restaurants, delivery platforms like UberEats, and meal planning applications.
    \item \textbf{Voice Input and Accessibility Enhancements: } Later iterations will incorporate voice recognition, assisted voice input, and screen reader compatibility for improved accessibility.
\end{enumerate}
% End SubSection

% End Section
\section{Use Case Diagram}
\label{sec:use_case_diagram}
% Begin Section
\begin{itemize}
	\item Provide the use case diagram for the system being developed.
	\item You do not need to provide the textual description of any of the use cases here (these will be specified under "Highlights of Functional Requirements").
%	\item Provide \emph{one} use case diagram for the most important Business Event.
%	\item The text of all use cases will be specified under "Highlights of Functional Requirements"
\end{itemize}
%In this section, select the most important Business Event that your system responds to and give its use case diagram.  Only one use case diagram is needed.  Give a brief textual description of the use case without repeating what is in the scenarios of the corresponding Business Event.

%
%
%
%This section should provide a use case diagram for your application. 
%\begin{enumerate}[a)]
%	\item Each use case appearing in the diagram should be accompanied by a text description. 
%\end{enumerate}
%% End Section

\section{Highlights of Functional Requirements}
\label{sec:functional_requirements}
% Begin Section
\begin{itemize}
	\item Specify all use cases (or other scenarios triggered by other events), organized by Business Event. 
	\item For each Business Event, show the scenario from every Viewpoint. You should have the same set of Viewpoints across all Business Events. If a Viewpoint doesn't participate, write N/A so we know you considered it still. You can choose how to present this - keep in mind it should be easy to follow. 
	\item At the end, combine them all into a Global Scenario.
	%\item Specify the "use cases" (or other triggering events) organized by Business Event. (The Global Scenario is what you might think of as a use case). Be sure to consider Business Events that aren't just triggered by users with goals (e.g. something happens in the environment that your system needs to respond to)
	\item Your focus should be on what the system needs to do, not how to do it. Specify it in enough detail that it clearly specifies what needs to be accomplished, but not so detailed that you start programming or making design decisions.
	\item Keep the length of each use case (Global Scenario) manageable. If it's getting too long, split into sub-cases.
	\item You are \emph{not} specifying a complete and consistent set of functional requirements here. (i.e. you are providing them in the form of use cases/global scenarios, not a refined list). For the purpose of this project, you do not need to reduce them to a list; the global scenarios format is all you need.
	\item Red text below is just to highlight where you need to insert a scenario - don't actually write it all in red.
\end{itemize}

\noindent {\bf Main Business Events:}
\begin{itemize}
	\item BE1 Inquiry of dish
	\item BE2 Create an account
	\item BE3 Login
	\item BE4 Add custom recipe
	\item BE5 Search food data
	\item BE6 Ask food recommendation
	\item BE7 Delete account
\end{itemize}

\noindent {\bf Viewpoints:}
\begin{itemize}
	\item VP1 User
	\item VP2 Health Canada
	\item VP3 Restaurant
	\item VP4 Security
	\item Customer Retention
\end{itemize}

\noindent {\bf Interpretation:} Specify any liberties you took in interpreting business events, if necessary.\\

\begin{enumerate}[{\bf BE1.}]
	% Business Event 1.
	\item Inquiry of dish \#1
	
	\textbf{Pre-Condition:} User opens the application; they have an account and are logged in. The user also has some details of the dish to inquire about.
		% Viewpoints
		\begin{enumerate}[{\bf VP1.}]
			\item User \#1 \\
				\textbf{Main Success Scenario:} 
				\begin{enumerate}[{1.}]
					\item User presses a button to inquire about a dish.
					\item System prompts user to provide a photo of the food.
					\item User uploads a picture from their gallary or takes a photo of the food.
					\item System stores the picture.
					\item System prompts the user to provide textual description of the food.
					\item User types in description of food and presses a button to continue.
					\item System stores the textual description.
					\item System prompts user to provide the ingredients found on the dish.
					\item User enters the ingredient(s) found in the dish and presses a button to continue.
					\item System stores the ingredient list.
					\item System identifies the dish and displays the name of dish to the user as well as nutritional information.
				\end{enumerate}
				\textbf{Secondary Scenario:}
				\begin{enumerate}
					\item[4.i.] User does not provide a photo.
					\begin{enumerate}
						\item[4.i.1.] System does not store any picture and continues to BE1.5.
					\end{enumerate}
					\item[4.ii.] User provides a blury picture.
					\begin{enumerate}
						\item[4.ii.1.] System warns user and ask user to try again.
					\end{enumerate}
				\end{enumerate}
				\begin{enumerate}
					\item[7.i.] User does not provide any textual description.
					\begin{enumerate}
						\item[7.i.1.] System stores an empty text and continues to BE1.6.
					\end{enumerate}
				\end{enumerate}
				\begin{enumerate}
					\item[10.i.] User does not provide any ingredient.
					\begin{enumerate}
						\item[10.i.1.] System stores an empty ingredient list and continues to BE1.11.
					\end{enumerate}
				\end{enumerate}
				\begin{enumerate}
					\item[11.i.] User did not provide either a picture, textual description, nor ingredient list
					\begin{enumerate}
						\item[11.i.1.] System does not accept inquiry request.
						\item[11.i.2.] System notifies user that no input data was given.
						\item[11.i.3.] System returns to home page.
					\end{enumerate}
					\item[11.ii.] System requests feedback of identified food.
					\begin{enumerate}
						\item[11.ii.1.] User enters feedback
						\item[11.ii.2.] System analyzes feedback and thanks user for feedback.
					\end{enumerate}
				\end{enumerate}

			\item Health Canada  \#2
				\begin{enumerate}
					\item[11.iii.] System must provide nutritional data in accordance with Health Canada's nutrition labelling policies \cite{CanadaNutrition}.
				\end{enumerate}
			\item Restaurant \#3
				\begin{enumerate}
					\item[11.iii.] System provides 5 star restaurants serving the returned dish.
				\end{enumerate}
			\item Security \#4 
				\begin{enumerate}
					\item[11.iv.] System delete user provided photo from memory.
				\end{enumerate}
			\item Customer Retention \#5
				\begin{enumerate}
					\item[11.v.] System saves identified dish in back-end data storage under user history.
				\end{enumerate}
		\end{enumerate}

		{\bf Global Scenario:} \\
		TODO: Combine to global scenario once team agrees with each viewpoint.

	% Business Event 2.
	\item Create an account \#2
	
	\textbf{Pre-Condition:} User has the mobile application downloaded on their android device and they do not have an account.
		% Viewpoints
		\begin{enumerate}[{\bf VP1.}]
			\item User \#1 \\
				\textbf{Main Success Scenario:} 
				\begin{enumerate}[{1.}]
					\item System prompts user to enter email address and password.
					\item User enters email and password for the app
					\item System verifies user email has not been registed and emails a verification code.
					\item User enters emailed verification code on the application.
					\item System verifies verification code.
					\item System creates an account and stores in database.
					\item System prompts user to enter any allergies or dietary restrictions.
					\item User enters allergis or dietary restrictions or skips.
					\item Systems opens the application's home page.
				\end{enumerate}
				\textbf{Secondary Scenario:}
				\begin{enumerate}
					\item[2.i.] User does not provide a valid password.
					\begin{enumerate}
						\item[2.i.1.] System does not accept the password and reminds user of password criteria.
					\end{enumerate}
				\end{enumerate}
				\begin{enumerate}
					\item[4.i.] Entered verificaion code is not correct.
					\begin{enumerate}
						\item[4.i.1.] System does not accept code.
						\item[4.i.2.] Create account failed.
					\end{enumerate}
				\end{enumerate}

			\item Health Canada  \#2 \\
				N/A
			\item Restaurant \#3 \\
				N/A
			\item Security \#4 
			\begin{enumerate}
				\item[4.ii.] Verification time limit has passed.
				\begin{enumerate}
					\item[4.ii.1.] System does not accept code.
					\item[4.ii.2.] Create account failed.
				\end{enumerate}
			\end{enumerate}
			\item Customer Retention \#5 \\
				N/A
		\end{enumerate}

		{\bf Global Scenario:} \\
		TODO: Combine to global scenario once team agrees with each viewpoint.

	% Business Event 3.
	\item Login \#3
	
	\textbf{Pre-Condition:} User opens the application and they have an account.
		% Viewpoints
		\begin{enumerate}[{\bf VP1.}]
			\item User \#1 \\
				\textbf{Main Success Scenario:} 
				\begin{enumerate}[{1.}]
					\item System prompts user to enter email and password.
					\item User enters login details.
					\item System authenticates login data.
					\item System brings user to home page.
				\end{enumerate}
				\textbf{Secondary Scenario:}
				\begin{enumerate}
					\item[2.i.] User forgets password.
					\begin{enumerate}
						\item[2.i.1.] User presses button to reset password.
						\item[2.i.2.] System sends password reset link to email the user provided if email is registered.
					\end{enumerate}
				\end{enumerate}
			\item Health Canada  \#2 \\
				N/A
			\item Restaurant \#3 \\
				N/A
			\item Security \#4 
			\begin{enumerate}
				\item[3.i.] System authentication fails.
				\begin{enumerate}
					\item[3.i.1.] System prompts warns user of invalid login.
					\item[3.i.2.] System permits retry of maximum 3 times before locking login feature for some time.
				\end{enumerate}
			\end{enumerate}
			\item Customer Retention \#5 \\
				N/A
		\end{enumerate}

		{\bf Global Scenario:} \\
		TODO: Combine to global scenario once team agrees with each viewpoint.

		% Business Event 4.
		\item Add custom recipe \#4
	
		\textbf{Pre-Condition:} User opens the application; they have an account and are logged in.
			% Viewpoints
			\begin{enumerate}[{\bf VP1.}]
				\item User \#1 \\
					\textbf{Main Success Scenario:} 
					\begin{enumerate}[{1.}]
						\item User presses button to add a recipe.
						\item System prompts to enter name of recipe.
						\item User enters name of recipe.
						\item System prompts serving size info.
						\item User enters serving size info.
						\item System prompts to enter ingredient list.
						\item User enters ingredient list with their protion amount.
						\item System prompts user to write steps to making the dish.
						\item User provides the steps to make the dish.
						\item System saves the dish in database and notifies user of successful save.
					\end{enumerate}
					\textbf{Secondary Scenario:}
					\begin{enumerate}
						\item[2.i.] User types in food that is not known to the system.
						\begin{enumerate}
							\item[2.i.1.] System notifies user that food does not exist in the database.
							\item[2.i.2.] System suggests user to add a custom recipe of the food they search, return to BE4.VP1.2.
						\end{enumerate}
					\end{enumerate}
				\item Health Canada  \#2
					\item \begin{enumerate}
						\item[10.i.] Provided recipe is poisonous or hazardous.
						\begin{enumerate}
							\item[10.i.1.] System rejects recipe and does not add to the database.
							\item[10.i.2.] System provides information to user of why recipe is rejected.
						\end{enumerate}
					\end{enumerate}
					\item 
				\item Restaurant \#3 \\
					N/A
				\item Security \#4 
					N/A
				\item Customer Retention \#5
					\begin{enumerate}
						\item[10.ii.] System should prompt an appreciation page.
					\end{enumerate}
			\end{enumerate}
	
			{\bf Global Scenario:} \\
			TODO: Combine to global scenario once team agrees with each viewpoint.

	% Business Event 5.
	\item Search food data \#5
	
	\textbf{Pre-Condition:} User opens the application; they have an account and are logged in.
		% Viewpoints
		\begin{enumerate}[{\bf VP1.}]
			\item User \#1 \\
				\textbf{Main Success Scenario:} 
				\begin{enumerate}[{1.}]
					\item User presses button to search a food.
					\item System provides a text box and prompts user to enter the food name.
					\item User types the food names and presses enter.
					\item System searches in database and returns nutritional information of the food and a recipe.
				\end{enumerate}
				\textbf{Secondary Scenario:}
				\begin{enumerate}
					\item[2.i.] User types in food that is not known to the system.
					\begin{enumerate}
						\item[2.i.1.] System notifies user that food does not exist in the database.
						\item[2.i.2.] System suggests user to add a custom recipe of the food they search, return to BE4.VP1.2.
					\end{enumerate}
				\end{enumerate}
			\item Health Canada  \#2
				\begin{enumerate}
					\item[4.i.] System must provide nutritional data in accordance with Health Canada's nutrition labelling policies \cite{CanadaNutrition}.
				\end{enumerate}
			\item Restaurant \#3
				\begin{enumerate}
					\item[4.ii.] User presses button to get restaurant recommendation.
					\begin{enumerate}
						\item[4.ii.1.] System asks user to enter a postal code to serach in.
						\item[4.ii.2.] User enters postal code.
						\item[4.ii.3.] System provides top restaurants within the provided location.
					\end{enumerate}
					
				\end{enumerate}
			\item Security \#4 
				N/A
			\item Customer Retention \#5
				\begin{enumerate}
					\item[3.iii.] User likes the page.
					\begin{enumerate}
						\item[3.iii.1.] System saves the searched food in back-end data storage under user favorites.
						\item[3.iii.2.] System provides some recommendations, return to BE6.VP.1.3.
					\end{enumerate}
				\end{enumerate}
		\end{enumerate}

		{\bf Global Scenario:} \\
		TODO: Combine to global scenario once team agrees with each viewpoint.

	% Business Event 6.
	\item Ask for food recommendation \#6	
	\textbf{Pre-Condition:} User opens the application; they have an account and are logged in.
		% Viewpoints
		\begin{enumerate}[{\bf VP1.}]
			\item User \#1 \\
				\textbf{Main Success Scenario:} 
				\begin{enumerate}[{1.}]
					\item User presses button for system to recommend food.
					\item System analyzes user data from database to recommend food. and stores result in user history.
					\item System displays the recommened food with pictures and text.
				\end{enumerate}
				\textbf{Secondary Scenario:}
				\begin{enumerate}
					\item[2.i.] System does not have any user data..
					\begin{enumerate}
						\item[2.i.1.] System randomly picks a food.
						\item[2.i.2.] System stores randomly chosen food in user history. 
					\end{enumerate}
				\end{enumerate}
			\item Health Canada  \#2
				\begin{enumerate}
					\item [2.ii.] System shall avoid recommending food that contains the user's allergies or is part of user's dietary restrictions.
				\end{enumerate}
			\item Restaurant \#3
				\begin{enumerate}
					\item[4.ii.] User presses button to get restaurant recommendation.
					\begin{enumerate}
						\item[4.ii.1.] System asks user to enter a postal code to serach in.
						\item[4.ii.2.] User enters postal code.
						\item[4.ii.3.] System provides top restaurants within the provided location.
					\end{enumerate}
				\end{enumerate}
			\item Security \#4 
				N/A
			\item Customer Retention \#5
				\begin{enumerate}
					\item[3.iii.] User likes the recommendation.
					\begin{enumerate}
						\item[3.iii.1.] System saves the recommended food in back-end data storage under user favorites.
						\item[3.iii.2.] System provides similar recommendations, return to BE6.VP.1.3.
					\end{enumerate}
				\end{enumerate}
		\end{enumerate}

		{\bf Global Scenario:} \\
		TODO: Combine to global scenario once team agrees with each viewpoint.

	% Business Event 7.
	\item Delete account \#6	
	\textbf{Pre-Condition:} User opens the application; they have an account and are logged in.
		% Viewpoints
		\begin{enumerate}[{\bf VP1.}]
			\item User \#1 \\
				\textbf{Main Success Scenario:} 
				\begin{enumerate}[{1.}]
					\item User presses button for system to delete account.
					\item System warns user that they are deleting their account.
					\item User confirms to delete the account.
					\item System prompts reason of deleting the account.
					\item User enters reason of deleting.
					\item System saves reason of deleting to data storage.
					\item System removes user account from data storage.
					\item System notifies user account is deleted and returns to create account page.
				\end{enumerate}
				\textbf{Secondary Scenario:}
				\begin{enumerate}
					\item[3.i.] User cancels deletion
					\begin{enumerate}
						\item[3.i.1.] System returns to home page. 
					\end{enumerate}
				\end{enumerate}
				\begin{enumerate}
					\item[5.i.] User cancels deletion
					\begin{enumerate}
						\item[5.i.1.] System returns to home page. 
					\end{enumerate}
					\item [5.ii] User skips providing reason of deletion.
					\begin{enumerate}
						\item[5.ii.1.] System does not store any data about reason of deletion.
					\end{enumerate}
				\end{enumerate}
			\item Health Canada  \#2 \\
				N/A
			\item Restaurant \#3 \\
				N/A
			\item Security \#4 
				\begin{enumerate}
					\item [6.i.] System shall save the reason of deleting account without associating the user to it.
					\item [7.i.] System shall not have any user data in any of its data storage systems.
				\end{enumerate}
			\item Customer Retention \#5
				\begin{enumerate}
					\item[4.i.] System shall put the user's favourite food on the feedback form.
				\end{enumerate}
		\end{enumerate}

		{\bf Global Scenario:} \\
		TODO: Combine to global scenario once team agrees with each viewpoint.
		
\end{enumerate}

%	Below, we organize by Business Event.
%	\begin{enumerate}[{BE}1.]
%		\item Business Event name
%		\begin{enumerate}[{VP1}.1]
%			\item Viewpoint name \newline
%			\noindent\fbox{%
%				\parbox{0.5\textwidth}{%
%					\begin{itemize}
%						\item {\bf $S_{1}$:} Initial response of the system to the Business Event
%						\item {\bf $E_{1}$:}  Reaction of the environment to $S_{1}$
%						\item {\bf $S_{2}$:}  Response of the system to $E_{1}$
%						\item {\bf $E_{2}$:}  Reaction of the environment to $S_{2}$
%						\item[] $\cdots$
%						\item {\bf $S_{n}$:}  Response of the system to $E_{(n-1)}$
%						\item {\bf $E_{n}$:}  Reaction of the environment to $E_{(n-1)}$
%						\item {\bf $S_{(n+1)}$:} Final response of the system concluding its function regarding the Business Event
%					\end{itemize}
%				}%
%			}
%			\item Viewpoint name\newline
%			\noindent\fbox{%
%				\parbox{0.5\textwidth}{%
%					\begin{itemize}
%						\item {\bf $S_{1}$:} Initial response of the system to the Business Event
%						\item {\bf $E_{1}$:}  Reaction of the environment to $S_{1}$
%						\item {\bf $S_{2}$:}  Response of the system to $E_{1}$
%						\item {\bf $E_{2}$:}  Reaction of the environment to $S_{2}$
%						\item[] $\cdots$
%						\item {\bf $S_{k}$:}  Response of the system to $E_{(k-1)}$
%						\item {\bf $E_{k}$:}  Reaction of the environment to $E_{(k-1)}$
%						\item {\bf $S_{(k+1)}$:} Final response of the system concluding its function regarding the Business Event
%					\end{itemize}
%				}%
%			}
%			\item \dots
%			\item \dots
%			\item \dots
%			\item[\dots]
%		\end{enumerate}	
%		\item[] {\bf Global Scenario of {\it Business Event Name}:} It is the scenario corresponding to the integration of all the above scenarios from the different Viewpoints of the Business Event BE1.\newline
%		\noindent\fbox{%
%			\parbox{0.5\textwidth}{%
%				\begin{itemize}
%					\item {\bf $S_{1}$:} Initial response of the system to the Business Event
%					\item {\bf $E_{1}$:}  Reaction of the environment to $S_{1}$
%					\item {\bf $S_{2}$:}  Response of the system to $E_{1}$
%					\item {\bf $E_{2}$:}  Reaction of the environment to $S_{2}$
%					\item[] $\cdots$
%					\item {\bf $S_{m}$:}  Response of the system to $E_{(m-1)}$
%					\item {\bf $E_{m}$:}  Reaction of the environment to $E_{(m-1)}$
%					\item {\bf $S_{(m+1)}$:} Final response of the system concluding its function regarding the Business Event
%				\end{itemize}
%			}%
%		}	
%		%\end{enumerate}
%		\item Business Event name
%		\begin{enumerate}[{VP1}.1]
%			\item Viewpoint name \newline
%			\noindent\fbox{%
%				\parbox{0.5\textwidth}{%
%					\begin{itemize}
%						\item {\bf $S_{1}$:} Initial response of the system to the Business Event
%						\item {\bf $E_{1}$:}  Reaction of the environment to $S_{1}$
%						\item {\bf $S_{2}$:}  Response of the system to $E_{1}$
%						\item {\bf $E_{2}$:}  Reaction of the environment to $S_{2}$
%						\item[] $\cdots$
%						\item {\bf $S_{n'}$:}  Response of the system to $E_{(n'-1)}$
%						\item {\bf $E_{n'}$:}  Reaction of the environment to $E_{(n'-1)}$
%						\item {\bf $S_{(n'+1)}$:} Final response of the system concluding its function regarding the Business Event
%					\end{itemize}
%				}%
%			}
%			\item Viewpoint name\newline
%			\noindent\fbox{%
%				\parbox{0.5\textwidth}{%
%					\begin{itemize}
%						\item {\bf $S_{1}$:} Initial response of the system to the Business Event
%						\item {\bf $E_{1}$:}  Reaction of the environment to $S_{1}$
%						\item {\bf $S_{2}$:}  Response of the system to $E_{1}$
%						\item {\bf $E_{2}$:}  Reaction of the environment to $S_{2}$
%						\item[] $\cdots$
%						\item {\bf $S_{k'}$:}  Response of the system to $E_{(k'-1)}$
%						\item {\bf $E_{k'}$:}  Reaction of the environment to $E_{(k'-1)}$
%						\item {\bf $S_{(k'+1)}$:} Final response of the system concluding its function regarding the Business Event
%					\end{itemize}
%				}%
%			}
%			\item \dots
%			\item \dots
%			\item \dots
%			\item[\dots]
%		\end{enumerate}	
%		\item[] {\bf Global Scenario of {\it Business Event Name}:} It is the scenario corresponding to the integration of all the above scenarios from the different Viewpoints of the Business Event BE2.\newline
%		\noindent\fbox{%
%			\parbox{0.5\textwidth}{%
%				\begin{itemize}
%					\item {\bf $S_{1}$:} Initial response of the system to the Business Event
%					\item {\bf $E_{1}$:}  Reaction of the environment to $S_{1}$
%					\item {\bf $S_{2}$:}  Response of the system to $E_{1}$
%					\item {\bf $E_{2}$:}  Reaction of the environment to $S_{2}$
%					\item[] $\cdots$
%					\item {\bf $S_{m'}$:}  Response of the system to $E_{(m'-1)}$
%					\item {\bf $E_{m'}$:}  Reaction of the environment to $E_{(m'-1)}$
%					\item {\bf $S_{(m'+1)}$:} Final response of the system concluding its function regarding the Business Event
%				\end{itemize}
%			}%
%		}		
%	\end{enumerate}

%End Section

\section{Non-Functional Requirements}
\label{sec:non-functional_requirements}


\begin{itemize}
	\item For each non-functional requirement, provide a justification/rationale for it.\\
	{\bf Example:} \\
	SC1. \emph{The device should not explode in a customer’s pocket.}\\
	{\bf Rationale:} Other companies have had issues with the batteries they used in their phones randomly exploding [insert citation]. This causes a safety issue, as the phone is often carried in a person's hand or pocket.	
	\item If you need to make a guess because you couldn't really talk to stakeholders, you can say "We imagined stakeholders would want...because..."
	\item Each requirement should have a unique label/number for it.
	\item In the list below, if a particular section doesn't apply, just write N/A so we know you considered it.
\end{itemize}

% Begin Section
\subsection{Look and Feel Requirements}
\label{sub:look_and_feel_requirements}
% Begin SubSection

\subsubsection{Appearance Requirements}
\label{ssub:appearance_requirements}
% Begin SubSubSection
\begin{enumerate}[{LF-A}1. ]
    \item The application shall use a visually appealing colour scheme
    \item The application shall have a high contrast between all text and background elements so users can easily read the text.
    \item No special prompt shall be required to display photos in application pages
\end{enumerate}
% End SubSubSection

\subsubsection{Style Requirements}
\label{ssub:style_requirements}
% Begin SubSubSection
\begin{enumerate}[{LF-S}1. ]
    \item The application shall use sans-serif for all its text, as studies indicate it is more readable than serif counterparts
    \item The application shall use the same font for the same text category.
    \item All text in the application shall conform to standard writing practices, including the ones for capitalization, spelling, grammar, bold text, italics, and underlining.
\end{enumerate}
% End SubSubSection

% End SubSection

\subsection{Usability and Humanity Requirements}
\label{sub:usability_and_humanity_requirements}
% Begin SubSection

\subsubsection{Ease of Use Requirements}
\label{ssub:ease_of_use_requirements}
% Begin SubSubSection
\begin{enumerate}[{UH-EOU}1. ]
    \item The application shall have an intuitive user interface  (UI) such that a user will grasp how to use the application almost instantly.
    \item The user experience (UX) shall be pleasant and incentivize users to return to the application.
\end{enumerate}
% End SubSubSection

\subsubsection{Personalization and Internationalization Requirements}
\label{ssub:personalization_and_internationalization_requirements}
% Begin SubSubSection
\begin{enumerate}[{UH-PI}1. ]
	\item The application shall be available both in light mode and dark mode, although one mode may be hidden behind a paywall.
    \item Users shall be able to create accounts and have their public activity on the application tied to their account.
\end{enumerate}
% End SubSubSection

\subsubsection{Learning Requirements}
\label{ssub:learning_requirements}
% Begin SubSubSection
\begin{enumerate}[{UH-L}1. ]
	\item Not applicable.
\end{enumerate}
% End SubSubSection

\subsubsection{Understandability and Politeness Requirements}
\label{ssub:understandability_and_politeness_requirements}
% Begin SubSubSection
\begin{enumerate}[{UH-UP}1. ]
	\item Responses shall be easily understood by the end user, including users that have a professional but not fluent understanding of English.
        \item Prompts should use polite phrases such as "please" when requesting and "thank you" after a user fulfills a request.
\end{enumerate}
% End SubSubSection

\subsubsection{Accessibility Requirements}
\label{ssub:accessibility_requirements}
% Begin SubSubSection
\begin{enumerate}[{UH-A}1. ]
	\item The application shall have a large text mode to accommodate users with severe nearsightedness.
    \item The application shall be compatible with screen readers for blind people.
\end{enumerate}
% End SubSubSection

% End SubSection

\subsection{Performance Requirements}
\label{sub:performance_requirements}
% Begin SubSection

\subsubsection{Speed and Latency Requirements}
\label{ssub:speed_and_latency_requirements}
% Begin SubSubSection
\begin{enumerate}[{PR-SL}1. ]
	\item The latency between sending a request for classification and getting the response, assuming a stable internet connection shall be no greater than five (5) minutes.
    \item The application's loading time shall not exceed thirty (30) seconds.
    \item The latency between changing windows within an application shall be negligible, assuming a stable internet connection.
\end{enumerate}
% End SubSubSection

\subsubsection{Safety-Critical Requirements}
\label{ssub:safety_critical_requirements}
% Begin SubSubSection
\begin{enumerate}[{PR-SC}1. ]
	\item 
\end{enumerate}
% End SubSubSection

\subsubsection{Precision or Accuracy Requirements}
\label{ssub:precision_or_accuracy_requirements}
% Begin SubSubSection
\begin{enumerate}[{PR-PA}1. ]
	\item All information given in the application shall be verifiable by subject matter experts or generally reliable sources.
\end{enumerate}
% End SubSubSection

\subsubsection{Reliability and Availability Requirements}
\label{ssub:reliability_and_availability_requirements}
% Begin SubSubSection
\begin{enumerate}[{PR-RA}1. ]
	\item The application shall be available 99\% of the time.
        \item The application shall inform users of any failure encountered, along with easily understandable advice to succeed on the next attempt. Usually, the advice would be to restart the application.
\end{enumerate}
% End SubSubSection

\subsubsection{Robustness or Fault-Tolerance Requirements}
\label{ssub:robustness_or_fault_tolerance_requirements}
% Begin SubSubSection
\begin{enumerate}[{PR-RFT}1. ]
	\item The application shall correct minor input and output errors when the intended meaning is obvious, including spelling.
\end{enumerate}
% End SubSubSection

\subsubsection{Capacity Requirements}
\label{ssub:capacity_requirements}
% Begin SubSubSection
\begin{enumerate}[{PR-C}1. ]
	\item No human experts shall be required to compliment the non-human experts when a user uploads a picture of a dish and data about it to achieve a response.
\end{enumerate}
% End SubSubSection

\subsubsection{Scalability or Extensibility Requirements}
\label{ssub:scalability_or_extensibility_requirements}
% Begin SubSubSection
\begin{enumerate}[{PR-SE}1. ]
	\item The application shall be able to handle one hundred (100) requests per second without any added latency.
\end{enumerate}
% End SubSubSection

\subsubsection{Longevity Requirements}
\label{ssub:longevity_requirements}
% Begin SubSubSection
\begin{enumerate}[{PR-L}1. ]
	\item The application shall last at least two years without any major updates, besides adding new functionality not in the initial release.
\end{enumerate}
% End SubSubSection

% End SubSection

\subsection{Operational and Environmental Requirements}
\label{sub:operational_and_environmental_requirements}
% Begin SubSection

\subsubsection{Expected Physical Environment}
\label{ssub:expected_physical_environment}
% Begin SubSubSection
\begin{enumerate}[{OE-EPE}1. ]
	\item This application shall be compatible with all internal databases and artificial intelligence used by the system's non-human experts. 
\end{enumerate}
% End SubSubSection

\subsubsection{Requirements for Interfacing with Adjacent Systems}
\label{ssub:requirements_for_interfacing_with_adjacent_systems}
% Begin SubSubSection
\begin{enumerate}[{OE-IA}1. ]
	\item This application shall be compatible with mobile devices running the Android platform.
        \item This application shall conform to the terms of service of the Android Application store.
        \item If a parallel iOS mobile application is developed, it shall conform to the terms of service of the Apple Application store.
    
\end{enumerate}
% End SubSubSection

\subsubsection{Productization Requirements}
\label{ssub:productization_requirements}
% Begin SubSubSection
\begin{enumerate}[{OE-P}1. ]
	\item Not applicable.
\end{enumerate}
% End SubSubSection

\subsubsection{Release Requirements}
\label{ssub:release_requirements}
% Begin SubSubSection
\begin{enumerate}[{OE-R}1. ]
	\item This application shall be released for Android devices by April 8, 2025.
\end{enumerate}
% End SubSubSection

% End SubSection

\subsection{Maintainability and Support Requirements}
\label{sub:maintainability_and_support_requirements}
% Begin SubSection

\subsubsection{Maintenance Requirements}
\label{ssub:maintenance_requirements}
% Begin SubSubSection
\begin{enumerate}[{MS-M}1. ]
	\item The system shall be designed for easy debugging and troubleshooting, with clear logs and error reports.
\end{enumerate}
% End SubSubSection

\subsubsection{Supportability Requirements}
\label{ssub:supportability_requirements}
% Begin SubSubSection
\begin{enumerate}[{MS-S}1. ]
	\item The application shall be compatible with at least the last three major versions of Android and iOS.
\end{enumerate}
% End SubSubSection

\subsubsection{Adaptability Requirements}
\label{ssub:adaptability_requirements}
% Begin SubSubSection
\begin{enumerate}[{MS-A}1. ]
	\item The system architecture shall support future expansion by allowing the addition of new expert modules without requiring significant refactoring.
\end{enumerate}
% End SubSubSection

% End SubSection

\subsection{Security Requirements}
\label{sub:security_requirements}
% Begin SubSection

\subsubsection{Access Requirements}
\label{ssub:access_requirements}
% Begin SubSubSection
\begin{enumerate}[{SR-AC}1. ]
	\item User authentication shall be required for storing identification history, ensuring only authorized users access their data.
\end{enumerate}
% End SubSubSection

\subsubsection{Integrity Requirements}
\label{ssub:integrity_requirements}
% Begin SubSubSection
\begin{enumerate}[{SR-INT}1. ]
	\item Data stored in the system shall be protected from unauthorized modification using cryptographic hashing and validation mechanisms.
\end{enumerate}
% End SubSubSection

\subsubsection{Privacy Requirements}
\label{ssub:privacy_requirements}
% Begin SubSubSection
\begin{enumerate}[{SR-P}1. ]
	\item The system shall allow users to delete their stored identification history on request.
\end{enumerate}
% End SubSubSection

\subsubsection{Audit Requirements}
\label{ssub:audit_requirements}
% Begin SubSubSection
\begin{enumerate}[{SR-AU}1. ]
	\item The system shall log all expert consultations and final decisions, including timestamps and user interactions, for auditing purposes.
\end{enumerate}
% End SubSubSection

\subsubsection{Immunity Requirements}
\label{ssub:immunity_requirements}
% Begin SubSubSection
\begin{enumerate}[{SR-IM}1. ]
	\item The system shall be resistant to SQL injection, cross-site scripting (XSS), and other common cyber threats.
\end{enumerate}
% End SubSubSection

% End SubSection

\subsection{Cultural and Political Requirements}
\label{sub:cultural_and_political_requirements}
% Begin SubSection

\subsubsection{Cultural Requirements}
\label{ssub:cultural_requirements}
% Begin SubSubSection
\begin{enumerate}[{CP-C}1. ]
	\item The application shall ensure that identifications are culturally sensitive and do not promote stereotypes or biases.
\end{enumerate}
% End SubSubSection

\subsubsection{Political Requirements}
\label{ssub:political_requirements}
% Begin SubSubSection
\begin{enumerate}[{CP-P}1. ]
	\item The application shall remain politically neutral and shall not classify or provide recommendations based on politically sensitive topics, ideologies, or national symbols.
\end{enumerate}
% End SubSubSection

% End SubSection

\subsection{Legal Requirements}
\label{sub:legal_requirements}
% Begin SubSection

\subsubsection{Compliance Requirements}
\label{ssub:compliance_requirements}
% Begin SubSubSection
\begin{enumerate}[{LR-COMP}1. ]
	\item The system shall comply with data protection laws such as GDPR, CCPA, and other applicable regulations regarding user data storage and processing.
\end{enumerate}
% End SubSubSection

\subsubsection{Standards Requirements}
\label{ssub:standards_requirements}
% Begin SubSubSection
\begin{enumerate}[{LR-STD}1. ]
	\item The application shall follow ISO/IEC 27001 security standards for data protection and secure communication.
\end{enumerate}
% End SubSubSection

% End SubSection

% End Section

\appendix
\section{Division of Labour}
\label{sec:division_of_labour}
% Begin Section
Include a Division of Labour sheet which indicates the contributions of each team member. This sheet must be signed by all team members.
% End Section

%\newpage
%\section*{IMPORTANT NOTES}
%\begin{itemize}
%	\item Be sure to include all sections of the template in your document regardless whether you have something to write for each or not
%	\begin{itemize}
%		\item If you do not have anything to write in a section, indicate this by the \emph{N/A}, \emph{void}, \emph{none}, etc.
%	\end{itemize}
%	\item Uniquely number each of your requirements for easy identification and cross-referencing
%	\item Highlight terms that are defined in Section~1.3 (\textbf{Definitions, Acronyms, and Abbreviations}) with \textbf{bold}, \emph{italic} or \underline{underline}
%	\item For Deliverable 1, please highlight, in some fashion, all (you may have more than one) creative and innovative features. Your creative and innovative features will generally be described in Section~2.2 (\textbf{Product Functions}), but it will depend on the type of creative or innovative features you are including.
%\end{itemize}


\end{document}
%------------------------------------------------------------------------------