\documentclass[]{article}

% Imported Packages
%------------------------------------------------------------------------------
\usepackage{amssymb}
\usepackage{amstext}
\usepackage{amsthm}
\usepackage{amsmath}
\usepackage{enumerate}
\usepackage{fancyhdr}
\usepackage[margin=1in]{geometry}
\usepackage{graphicx}
%\usepackage{extarrows}
%\usepackage{setspace}
\usepackage{color}
\usepackage{float}
%------------------------------------------------------------------------------

% Header and Footer
%------------------------------------------------------------------------------
\pagestyle{plain}  
\renewcommand\headrulewidth{0.4pt}                                      
\renewcommand\footrulewidth{0.4pt}                                    
%------------------------------------------------------------------------------

% Title Details
%------------------------------------------------------------------------------
\title{Deliverable \#2: What’s That Dish Software}
\author{SE 3A04: Software Design II -- Large System Design}
\date{}                           
%------------------------------------------------------------------------------

% Document
%------------------------------------------------------------------------------
\begin{document}

\maketitle	
\noindent{\bf Tutorial Number:} T03\\
{\bf Group Number:} G03 \\
{\bf Group Members:} 
\begin{itemize}
	\item Imran Chowdhury
	\item Michael Roberts
	\item Sathurshan Arulmohan
	\item Tanisha Tasnin
	\item Zifan Si
\end{itemize}

\section*{IMPORTANT NOTES}
\begin{itemize}
	%	\item You do \underline{NOT} need to provide a text explanation of each diagram; the diagram should speak for itself
	\item Please document any non-standard notations that you may have used
	\begin{itemize}
		\item \emph{Rule of Thumb}: if you feel there is any doubt surrounding the meaning of your notations, document them
	\end{itemize}
	\item Some diagrams may be difficult to fit into one page
	\begin{itemize}
		\item Ensure that the text is readable when printed, or when viewed at 100\% on a regular laptop-sized screen.
		\item If you need to break a diagram onto multiple pages, please adopt a system of doing so and thoroughly explain how it can be reconnected from one page to the next; if you are unsure about this, please ask about it
	\end{itemize}
	\item Please submit the latest version of Deliverable 1 with Deliverable 2
	\begin{itemize}
		\item Indicate any changes you made.
	\end{itemize}
	\item If you do \underline{NOT} have a Division of Labour sheet, your deliverable will \underline{NOT} be marked
\end{itemize}

\newpage
\section{Introduction}
\label{sec:introduction}
% Begin Section

\subsection{Purpose}
\label{sub:purpose}
% Begin SubSection
This document provides a highlevel overview of the system architecture for the What’s That Dish mobile application.
It provides insight into the classes required to build the application, along with the chosen architectural style for the system and its subsystems.

This document is intended for internal stakeholders of What’s That Dish, which includes the software architects, software developers, QA testers, and the project manager.

It is recommended that Deliverable 1 to be reviewed in advance to gain a clear understanding of the application’s use cases, as well as its functional and non-functional requirements.
% End SubSection

\subsection{System Description}
\label{sub:system_description}
% Begin SubSection
Give a brief description of the system. This could be a paragraph or two to give some context to this document.

% End SubSection

\subsection{Overview}
\label{sub:overview}
% Begin SubSection
Section 2 presents the Analysis Class diagram for What's That Dish application.
Section 3 outlines the architectural design covering both the system architecture and the subsytems.
Section 4 provides Class Responsibility Collaboration cards for each class defined in the Analysis Class diagram.
% End SubSection

% End Section

\section{Analysis Class Diagram}
\label{sec:analysis_class_diagram}
% Begin Section
\begin{figure}[H]
	\centering
   \includegraphics[width=\textwidth]{image/D2_2_analysis_class_diagram.png}
\end{figure}
% End Section


\section{Architectural Design}
\label{sec:architectural_design}
% Begin Section
This section should provide an overview of the overall architectural design of your application. Your overall architecture should show the division of the system into subsystems with high cohesion and low coupling.

\subsection{System Architecture}
\label{sub:system_architecture}
% Begin SubSection
\begin{itemize}
	\item Identify and explain the overall architecture of your system
	\item Be sure to clearly state the name of the architecture you used (this is the name of the architectural pattern, not the name of your system)
	\item Provide the reasoning and justification of the choice of architecture
	\item Provide a structural architecture diagram showing the relationship among the subsystems (if appropriate)
	\item List any design alternatives you considered, but eliminated (and explain why you eliminated them)
\end{itemize}
% End SubSection

\subsection{Subsystems}
\label{sub:subsystems}
% Begin SubSection
\subsection{Subsystems}

The \textit{What’s That Dish} system is composed of multiple subsystems that work together to provide dish recommendations, manage user accounts, and process user inputs. Each subsystem operates within a loosely coupled architecture to ensure modularity, maintainability, and scalability.

\subsubsection{Account Management Subsystem}
The Account Management subsystem is responsible for handling user authentication, account creation, deletion, and profile management. It provides functionalities such as logging in, extracting user data, and managing stored profiles. The system maintains user information within the Account Database and ensures secure interactions with other subsystems.

\textbf{Collaborates with:} Create Account (Boundary), Delete Account (Boundary), Like Recommendation (Boundary), Ask Recommendation (Boundary).

\subsubsection{Recommendation Subsystem}
The Recommendation Controller subsystem provides dish recommendations based on user preferences, past interactions, and restaurant availability. It updates internal user recommendation history, tracks liked dishes, and communicates with restaurant databases to find suitable dining options.

\textbf{Collaborates with:} Like Recommendation (Boundary), Ask Recommendation (Boundary), Restaurant Database, User History Database.

\subsubsection{Dish Determination Subsystem}
The Dish Determination subsystem manages the process of dish identification based on user input. It collects dish-related data, processes it using various analytical techniques, and returns the final dish identification results to the Dish Inquiry Boundary. It acts as a central hub, controlling the analysis of dish-related data.

\textbf{Collaborates with:} Dish Inquiry (Boundary), Image Recognition, Ingredients Analysis, Text Analysis.

\subsubsection{Image Recognition Subsystem}
The Image Recognition subsystem enables dish identification through image analysis. It uses a trained machine learning model to predict the dish name from an uploaded picture. The Food Picture Database is utilized to improve the accuracy of dish predictions.

\textbf{Collaborates with:} Dish Determination, Food Picture Database.

\subsubsection{Ingredients Analysis Subsystem}
The Ingredients Analysis subsystem predicts dish names based on ingredient lists provided by users. It cross-references user-inputted ingredients with recipes in the Recipe Database to identify potential dish matches.

\textbf{Collaborates with:} Dish Determination, Recipe Database.

\subsubsection{Text Analysis Subsystem}
The Text Analysis subsystem allows users to identify dishes based on textual descriptions. It performs semantic analysis on user-inputted text to predict the dish name accurately. The Dish Database stores structured dish information, which is used for comparisons.

\textbf{Collaborates with:} Dish Determination, Dish Database.

\subsubsection{Recipe Management Subsystem}
The Recipe Management subsystem allows users to add and manage custom recipes. It ensures the proper organization and retrieval of user-submitted and system-generated recipes, storing all information in the Recipe Database.

\textbf{Collaborates with:} Add Recipe (Boundary), Recipe Database.

% End SubSection

% End Section
	
\section{Class Responsibility Collaboration (CRC) Cards}
\label{sec:class_responsibility_collaboration_crc_cards}
% Begin Section
This section should contain all of your CRC cards.

\begin{table}[H]
	\centering
	\begin{tabular}{|p{7cm}|p{7cm}|}
	\hline 
	 \multicolumn{2}{|l|}{\textbf{Class Name:}} \\
	\hline
	\textbf{Responsibility:} & \textbf{Collaborators:} \\
	\hline
	\raggedright
	\begin{itemize}
		\item TODO
	\end{itemize}
	\vspace{1in} & 
	\begin{itemize}
		\item TODO
	\end{itemize} \\
	\hline
	\end{tabular}
\end{table}

\begin{table}[H]
	\centering
	\begin{tabular}{|p{7cm}|p{7cm}|}
	\hline 
	 \multicolumn{2}{|l|}{\textbf{Class Name:} Create Account (Boundary)} \\
	\hline
	\textbf{Responsibility:} & \textbf{Collaborators:} \\
	\hline
	\raggedright
	\begin{itemize}
		\item Knows Acount Management
		\item Creates accounts
	\end{itemize}
	\vspace{1in} & 
	\begin{itemize}
		\item Account Management
	\end{itemize} \\
	\hline
	\end{tabular}
\end{table}

\begin{table}[H]
	\centering
	\begin{tabular}{|p{7cm}|p{7cm}|}
	\hline 
	 \multicolumn{2}{|l|}{\textbf{Class Name:} Delete account (Boundary)} \\
	\hline
	\textbf{Responsibility:} & \textbf{Collaborators:} \\
	\hline
	\raggedright
	\begin{itemize}
		\item Knows Acount Management
		\item Handles the event of a user deleting their account
	\end{itemize}
	\vspace{1in} & 
	\begin{itemize}
		\item Account Management
	\end{itemize} \\
	\hline
	\end{tabular}
\end{table}

\begin{table}[H]
	\centering
	\begin{tabular}{|p{7cm}|p{7cm}|}
	\hline 
	 \multicolumn{2}{|l|}{\textbf{Class Name:} Like Recommendation (Boundary)} \\
	\hline
	\textbf{Responsibility:} & \textbf{Collaborators:} \\
	\hline
	\raggedright
	\begin{itemize}
		\item Knows Recommendation Controller
		\item Handles the "click" event of a user pressing the "Like" button
	\end{itemize}
	\vspace{1in} & 
	\begin{itemize}
		\item Account Management
	\end{itemize} \\
	\hline
	\end{tabular}
\end{table}

\begin{table}[H]
	\centering
	\begin{tabular}{|p{7cm}|p{7cm}|}
	\hline 
	 \multicolumn{2}{|l|}{\textbf{Class Name:} Ask Recommendation (Boundary)} \\
	\hline
	\textbf{Responsibility:} & \textbf{Collaborators:} \\
	\hline
	\raggedright
	\begin{itemize}
		\item Knows Recommendation Controller
		\item Prompts a restaraunt user for recommendations (?)
	\end{itemize}
	\vspace{1in} & 
	\begin{itemize}
		\item Account Management
	\end{itemize} \\
	\hline
	\end{tabular}
\end{table}

\begin{table}[H]
	\centering
	\begin{tabular}{|p{7cm}|p{7cm}|}
	\hline 
	 \multicolumn{2}{|l|}{\textbf{Class Name:} Dish Inquiry (Boundary)} \\
	\hline
	\textbf{Responsibility:} & \textbf{Collaborators:} \\
	\hline
	\raggedright
	\begin{itemize}
		\item Knows Dish Determination
		\item Handles the event of a user inputting dish information and the system outputting results to the user
	\end{itemize}
	\vspace{1in} & 
	\begin{itemize}
		\item Dish Determination
	\end{itemize} \\
	\hline
	\end{tabular}
\end{table}

\begin{table}[H]
	\centering
	\begin{tabular}{|p{7cm}|p{7cm}|}
	\hline 
	 \multicolumn{2}{|l|}{\textbf{Class Name:} Add Recipe (Boundary)} \\
	\hline
	\textbf{Responsibility:} & \textbf{Collaborators:} \\
	\hline
	\raggedright
	\begin{itemize}
		\item Knows Recipe Managmenent
		\item Handles user request to add recipe
	\end{itemize}
	\vspace{1in} & 
	\begin{itemize}
		\item Dish Determination
	\end{itemize} \\
	\hline
	\end{tabular}
\end{table}

% Controller Classes
\begin{table}[H]
	\centering
	\begin{tabular}{|p{7cm}|p{7cm}|}
	\hline 
	 \multicolumn{2}{|l|}{\textbf{Class Name: Recommendation Controller}} \\
	\hline
	\textbf{Responsibility:} & \textbf{Collaborators:} \\
	\hline
	\raggedright
	\begin{itemize}
		\item Provides recommendation of dishes
		\item  Provides restarants serving recommended dish
		\item Updates internal user recommendation history
		\item Updates user liked dishes
	\end{itemize}
	\vspace{1in} & 
	\begin{itemize}
		\item Like Recommendation Boundary
		\item Ask Recommendation Boundary
		\item Restaurant Database
		\item User History Database
	\end{itemize} \\
	\hline
	\end{tabular}
\end{table}

\begin{table}[H]
	\centering
	\begin{tabular}{|p{7cm}|p{7cm}|}
	\hline 
	 \multicolumn{2}{|l|}{\textbf{Class Name: Account Management}} \\
	\hline
	\textbf{Responsibility:} & \textbf{Collaborators:} \\
	\hline
	\raggedright
	\begin{itemize}
		\item Manage accounts
		\item Log into user profile
		\item Extract user data
	\end{itemize}
	\vspace{1in} & 
	\begin{itemize}
		\item Account Database
	\end{itemize} \\
	\hline
	\end{tabular}
\end{table}

\begin{table}[H]
	\centering
	\begin{tabular}{|p{7cm}|p{7cm}|}
	\hline 
	 \multicolumn{2}{|l|}{\textbf{Class Name: Dish Determination}} \\
	\hline
	\textbf{Responsibility:} & \textbf{Collaborators:} \\
	\hline
	\raggedright
	\begin{itemize}
		\item Manage the determination of a dish process
		\item Control the clients that determine the dish
		\item Accept dish data
		\item Provide dish information to Dish Inquiry Page
	\end{itemize}
	\vspace{1in} & 
	\begin{itemize}
		\item Dish Inquiry Page
		\item Image Recognition
		\item Ingredients Analysis
		\item Text Analysis
	\end{itemize} \\
	\hline
	\end{tabular}
\end{table}

\begin{table}[H]
	\centering
	\begin{tabular}{|p{7cm}|p{7cm}|}
	\hline 
	 \multicolumn{2}{|l|}{\textbf{Class Name: Image Recognition}} \\
	\hline
	\textbf{Responsibility:} & \textbf{Collaborators:} \\
	\hline
	\raggedright
	\begin{itemize}
		\item Predict dish from picture
	\end{itemize}
	\vspace{1in} & 
	\begin{itemize}
		\item Dish Determination
		\item Food Picture Database
	\end{itemize} \\
	\hline
	\end{tabular}
\end{table}

\begin{table}[H]
	\centering
	\begin{tabular}{|p{7cm}|p{7cm}|}
	\hline 
	 \multicolumn{2}{|l|}{\textbf{Class Name: Ingredients Analysis}} \\
	\hline
	\textbf{Responsibility:} & \textbf{Collaborators:} \\
	\hline
	\raggedright
	\begin{itemize}
		\item Predict dish from ingredients list.
	\end{itemize}
	\vspace{1in} & 
	\begin{itemize}
		\item Dish Determination
		\item Recipe Database
	\end{itemize} \\
	\hline
	\end{tabular}
\end{table}

\begin{table}[H]
	\centering
	\begin{tabular}{|p{7cm}|p{7cm}|}
	\hline 
	 \multicolumn{2}{|l|}{\textbf{Class Name: Text Analysis}} \\
	\hline
	\textbf{Responsibility:} & \textbf{Collaborators:} \\
	\hline
	\raggedright
	\begin{itemize}
		\item Predict dish from text semantic analysis
	\end{itemize}
	\vspace{1in} & 
	\begin{itemize}
		\item Dish Determination
		\item Dish Database
	\end{itemize} \\
	\hline
	\end{tabular}
\end{table}

\begin{table}[H]
	\centering
	\begin{tabular}{|p{7cm}|p{7cm}|}
	\hline 
	 \multicolumn{2}{|l|}{\textbf{Class Name: Recipe Management}} \\
	\hline
	\textbf{Responsibility:} & \textbf{Collaborators:} \\
	\hline
	\raggedright
	\begin{itemize}
		\item Manage the Recipe Database.
		\item Accept new recipes
	\end{itemize}
	\vspace{1in} & 
	\begin{itemize}
		\item Add Custom Recipe Boundary
		\item Recipe Database
	\end{itemize} \\
	\hline
	\end{tabular}
\end{table}


% End Section

\appendix
\section{Division of Labour}
\label{sec:division_of_labour}
\textbf{Imran Chowdhury:}
\begin{enumerate}
	\item TODO
\end{enumerate}

\textbf{Signature:} Imran Chowdhury \\

\textbf{Michael Roberts:}
\begin{enumerate}
	\item TODO
\end{enumerate}

\begin{figure}[H]
 	\centering
    \includegraphics[width=\textwidth]{image/A_Michael_Roberts_Signature.png}
\end{figure}

\textbf{Sathurshan Arulmohan:}
\begin{enumerate}
	\item Brainstorming and drawing the Analysis Class Diagram.
	\item Wrote section 1.1 and 1.3.
	\item Wrote CRC for controller classes
\end{enumerate}

\textbf{Signature:} SATHURSHAN ARULMOHAN \\

\textbf{Tanisha Tasnin:}
\begin{enumerate}
	\item TODO
\end{enumerate}

\textbf{Signature:} TANISHA TASNIN \\

\textbf{Zifan Si:}
\begin{enumerate}
	\item TODO
\end{enumerate}

\textbf{Signature:} ZIFAN SI  \\


\end{document}
%------------------------------------------------------------------------------